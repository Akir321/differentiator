\documentclass{article}
\usepackage[russian]{babel}

\title{Вычисление производной}
\author{Авторы статьи возжелали остаться в тени}
\date{Время - иллюзия}

\begin{document}

\maketitle

\section{Вступление}

У современного человека в жизни часто возникает потребность дифференцировать. Бывает, что человек не может даже сопротивляться этому порыву: дай продифференцировать что-то или умру! Главная проблема заключается в том, что не каждый человек от рождения умеет вычислять производные функций. Авторы статьи решили помочь обездоленным и предоставить пример дифференцирования одной из функций, с помощью которого любой, даже самый отпет гуманитарий, смог бы научиться брать простенькие производные.

Кроме того, данная статья поясняет одно из важных приложений производных -- исследование функции. Вычисление производных является необходимой и важнейшей задачей при исследовании функции и построении её графика. При известных первой и второй производных читатель с легкостью сможет построить график функции, а данная статья научить эти производные искать.

Несмотря на свои добрые побуждения, авторы статьи не могут знать, к чему приведет ее использование. Авторы статьи не несут ответсвенности за последствия. Всё на вашей совести.

\section{Вычисление первой производной}

Давайте разберем такой несложный пример. Рассмотрим функцию $f(x) = $ $x ^ {x} + 5 \cdot x$ и найдем ее производную.

Если бы вас поставили перед выбором, взять производную или умереть, что бы вы выбрали?

$5 \cdot x$ $\Rightarrow$ $0 \cdot x + 5 \cdot 1$

Один из авторов статьи решил, что вычислить эту производную довольно легко. Другие не согласны -- это же тривиально!

$ \ln x$ $\Rightarrow$ $\frac{1}{x}$

Если бы вас поставили перед выбором, взять производную или умереть, что бы вы выбрали?

$x \cdot  \ln x$ $\Rightarrow$ $1 \cdot  \ln x + x \cdot \frac{1}{x}$

Конец близок, он где-то в этой производной:

$x ^ {x}$ $\Rightarrow$ $x ^ {x} \cdot (1 \cdot  \ln x + x \cdot \frac{1}{x})$

Один из авторов статьи решил, что вычислить эту производную довольно легко. Другие не согласны -- это же тривиально!

$x ^ {x} + 5 \cdot x$ $\Rightarrow$ $x ^ {x} \cdot (1 \cdot  \ln x + x \cdot \frac{1}{x}) + 0 \cdot x + 5 \cdot 1$

Очевидно, что на данное выражение без слез не взглянешь. Авторы считают, что его необходимо упростить. Путем тривиальных преобразований получим:

$f'(x) = $ $x ^ {x} \cdot ( \ln x + x \cdot \frac{1}{x}) + 5$\section{Вычисление второй производной}

Легко понять, что вычисление второй производной -- дело не менее важное, чем вычисление первой. По определению вторая производная суть производная производной, исходя из чего некоторые авторы статьи посчитали, что читатель, дошедший до этого места, без труда вычислит вторую производную самостоятельно. Всё же, в ходе всенародного голосования единогласно было принято решение привести в данной статье нахождение второй производной заданной функции.

Очевидно, что

$\frac{1}{x}$ $\Rightarrow$ $\frac{0 \cdot x - 1 \cdot 1}{x \cdot x}$

Ещё в советских яслях проходили, что

$x \cdot \frac{1}{x}$ $\Rightarrow$ $1 \cdot \frac{1}{x} + x \cdot \frac{0 \cdot x - 1 \cdot 1}{x \cdot x}$

Если бы вас поставили перед выбором, взять производную или умереть, что бы вы выбрали?

$ \ln x$ $\Rightarrow$ $\frac{1}{x}$

Что ещё уготовит на судьба?

$ \ln x + x \cdot \frac{1}{x}$ $\Rightarrow$ $\frac{1}{x} + 1 \cdot \frac{1}{x} + x \cdot \frac{0 \cdot x - 1 \cdot 1}{x \cdot x}$

Конец близок, он где-то в этой производной:

$ \ln x$ $\Rightarrow$ $\frac{1}{x}$

Если бы вас поставили перед выбором, взять производную или умереть, что бы вы выбрали?

$x \cdot  \ln x$ $\Rightarrow$ $1 \cdot  \ln x + x \cdot \frac{1}{x}$

Бери от жизни все, даже производную:

$x ^ {x}$ $\Rightarrow$ $x ^ {x} \cdot (1 \cdot  \ln x + x \cdot \frac{1}{x})$

Авторы статьи сомневаются в её необходимости. Это же и так очевидно:

$x ^ {x} \cdot ( \ln x + x \cdot \frac{1}{x})$ $\Rightarrow$ $x ^ {x} \cdot (1 \cdot  \ln x + x \cdot \frac{1}{x}) \cdot ( \ln x + x \cdot \frac{1}{x}) + x ^ {x} \cdot (\frac{1}{x} + 1 \cdot \frac{1}{x} + x \cdot \frac{0 \cdot x - 1 \cdot 1}{x \cdot x})$

Разделяй и властвуй!

$x ^ {x} \cdot ( \ln x + x \cdot \frac{1}{x}) + 5$ $\Rightarrow$ $x ^ {x} \cdot (1 \cdot  \ln x + x \cdot \frac{1}{x}) \cdot ( \ln x + x \cdot \frac{1}{x}) + x ^ {x} \cdot (\frac{1}{x} + 1 \cdot \frac{1}{x} + x \cdot \frac{0 \cdot x - 1 \cdot 1}{x \cdot x}) + 0$

С помощью нехитрых преобразований можем привести вторую производную к виду:

$f''(x) = $ $x ^ {x} \cdot ( \ln x + x \cdot \frac{1}{x}) \cdot ( \ln x + x \cdot \frac{1}{x}) + x ^ {x} \cdot (\frac{1}{x} + \frac{1}{x} + x \cdot \frac{-1}{x \cdot x})$

\section{Размышления об исследовании функции}

Теперь, когда вычислены первая и вторая производная, дальнейшие действия по исследованию функции тривиальны. Авторы статьи не считают необходимым приводить график заданной функции, ведь мудрый и увлеченный читатель уже сам давно его построил. Для тех же читателей, что еще не постоили график, эта задача предлагается в качестве упражнения.

Для закрепления знаний, связанных с нахождением производной, полученных в данной статье, авторы статьи предлагают читателю вычислить первую, вторую, третью, четвертую и пятую производные в точке $x_0$ в качестве упражнения. Зная эти производные, можно получить разложение функции по формуле Тейлора в окрестности данной точки. Авторы статьи надеятся, что данное упражнение не вызовет у читателя затруднений.

Формула Тейлор имеет следующий вид:
$$f(x) = f(x_0) + \frac{f'(x_0)}{1!}(x - x_0) + \cdots + \frac{f^{(n)}(x_0)}{n!}(x - x_0)^n + o((x - x_0)^n).$$

Чтобы упростить читателю задачу, авторы статьи вычислили первую и вторую производную в точке 0.



 $f(x) = -1.11111e+07 + -1.11111e+07x + -1.11111e+07x^2 + o(x^2)$ \section*{Источники}
\begin{enumerate}
	\item Любовь к математике;
	\item Горящие сердца авторов статьи;
	\item Божественные знания, данные нам Матом и Аном.
\end{enumerate}

\end{document}

