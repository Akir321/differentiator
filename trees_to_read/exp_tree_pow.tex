\documentclass{article}
\usepackage[russian]{babel}

\title{Вычисление производной}
\author{Авторы статьи возжелали остаться в тени}
\date{Время - иллюзия}

\begin{document}

\maketitle

\section{Вступление}

У современного человека в жизни часто возникает потребность дифференцировать. Бывает, что человек не может даже сопротивляться этому порыву: дай продифференцировать что-то или умру! Главная проблема заключается в том, что не каждый человек от рождения умеет вычислять производные функций. Авторы статьи решили помочь обездоленным и предоставить пример дифференцирования одной из функций, с помощью которого любой, даже самый отпет гуманитарий, смог бы научиться брать простенькие производные.

Несмотря на свои добрые побуждения, авторы статьи не могут знать, к чему приведет ее использование. Авторы статьи не несут ответсвенности за последствия. Всё на вашей совести.

\section{Вычисление производной}


Давайте разберем такой несложный пример. Рассмотрим функцию $x ^ {x} + 5 \cdot x$ и найдем ее производную.

Бери от жизни все, даже производную:

$5 \cdot x$ $\Rightarrow$ $0 \cdot x + 5 \cdot 1$

Производящее хозяйство нынче не в моде. Произведем хотя бы функцию

$ \ln x$ $\Rightarrow$ $\frac{1}{x}$

Любой школьник знает, что

$x \cdot  \ln x$ $\Rightarrow$ $1 \cdot  \ln x + x \cdot \frac{1}{x}$

Вычислим производную возведения в степень

$x ^ {x}$ $\Rightarrow$ $x ^ {x} \cdot (1 \cdot  \ln x + x \cdot \frac{1}{x})$

Конец близок, он где-то в этой производной:

$x ^ {x} + 5 \cdot x$ $\Rightarrow$ $x ^ {x} \cdot (1 \cdot  \ln x + x \cdot \frac{1}{x}) + 0 \cdot x + 5 \cdot 1$

Очевидно, что на данное выражение без слез не взглянешь. Авторы считают, что его необходимо упростить. Путем тривиальных преобразований получим:


$x ^ {x} \cdot ( \ln x + x \cdot x) + 5$\section{Размышления об исследовании функции}

Для закрепления знаний, полученных в данной статье, авторы статьи предлагают читателю вычислить первую, вторую, третью, четвертую и пятую производные в точке $x_0$ в качестве упражнения. Зная эти производные, можно получить разложение функции по формуле Тейлора в окрестности данной точки. Авторы статьи надеятся, что данное упражнение не вызовет у читателя затруднений.

Формула Тейлор имеет следующий вид:
$$f(x) = f(x_0) + \frac{f'(x_0)}{1!}(x - x_0) + \cdots + \frac{f^{(n)}(x_0)}{n!}(x - x_0)^n + o((x - x_0)^n).$$

Чтобы упростить читателю задачу, авторы статьи вычислили первую и вторую производную в точке 0.



\section*{Источники}
\begin{enumerate}
	\item Любовь к математике;
	\item Горящие сердца авторов статьи;
	\item Божественные знания, данные нам Матом и Аном.
\end{enumerate}

\end{document}

