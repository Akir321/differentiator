\documentclass{article}
\usepackage[russian]{babel}

\title{Вычисление производной}
\author{Авторы статьи возжелали остаться в тени}
\date{Время - иллюзия}

\begin{document}

\maketitle

\section{Вступление}

У современного человека в жизни часто возникает потребность дифференцировать. Бывает, что человек не может даже сопротивляться этому порыву: дай продифференцировать что-то или умру! Главная проблема заключается в том, что не каждый человек от рождения умеет вычислять производные функций. Авторы статьи решили помочь обездоленным и предоставить пример дифференцирования одной из функций, с помощью которого любой, даже самый отпет гуманитарий, смог бы научиться брать простенькие производные.

Кроме того, данная статья поясняет одно из важных приложений производных -- исследование функции. Вычисление производных является необходимой и важнейшей задачей при исследовании функции и построении её графика. При известных первой и второй производных читатель с легкостью сможет построить график функции, а данная статья научить эти производные искать.

Несмотря на свои добрые побуждения, авторы статьи не могут знать, к чему приведет ее использование. Авторы статьи не несут ответсвенности за последствия. Всё на вашей совести.

\section{Вычисление первой производной}

Давайте разберем такой несложный пример. Рассмотрим функцию $f(x) = $ $\log_{6 \cdot x}(5 \cdot x - 6) - 5$ и найдем ее производную.

Вычислим производную данного выражения:

$6 \cdot x$ $\Rightarrow$ $0 \cdot x + 6 \cdot 1$

Что ещё уготовит на судьба?

$ \ln (6 \cdot x)$ $\Rightarrow$ $\frac{0 \cdot x + 6 \cdot 1}{6 \cdot x}$

Авторы статьи сомневаются в её необходимости. Это же и так очевидно:

$5 \cdot x$ $\Rightarrow$ $0 \cdot x + 5 \cdot 1$

Любой школьник знает, что

$5 \cdot x - 6$ $\Rightarrow$ $(0 \cdot x + 5 \cdot 1) - 0$

Ещё в советских яслях проходили, что

$ \ln (5 \cdot x - 6)$ $\Rightarrow$ $\frac{(0 \cdot x + 5 \cdot 1) - 0}{5 \cdot x - 6}$

ТОЛЬКО НЕ ДЕЛЕНИЕ НА 0 ТОЛЬКО НЕ ДЕЛЕНИЕ НА 0

$\frac{ \ln (5 \cdot x - 6)}{ \ln (6 \cdot x)}$ $\Rightarrow$ $\frac{\frac{(0 \cdot x + 5 \cdot 1) - 0}{5 \cdot x - 6} \cdot  \ln (6 \cdot x) -  \ln (5 \cdot x - 6) \cdot \frac{0 \cdot x + 6 \cdot 1}{6 \cdot x}}{ \ln (6 \cdot x) \cdot  \ln (6 \cdot x)}$

Ещё в советских яслях проходили, что

$\log_{6 \cdot x}(5 \cdot x - 6)$ $\Rightarrow$ $\frac{\frac{(0 \cdot x + 5 \cdot 1) - 0}{5 \cdot x - 6} \cdot  \ln (6 \cdot x) -  \ln (5 \cdot x - 6) \cdot \frac{0 \cdot x + 6 \cdot 1}{6 \cdot x}}{ \ln (6 \cdot x) \cdot  \ln (6 \cdot x)}$

Вычислим производную разности

$\log_{6 \cdot x}(5 \cdot x - 6) - 5$ $\Rightarrow$ $\frac{\frac{(0 \cdot x + 5 \cdot 1) - 0}{5 \cdot x - 6} \cdot  \ln (6 \cdot x) -  \ln (5 \cdot x - 6) \cdot \frac{0 \cdot x + 6 \cdot 1}{6 \cdot x}}{ \ln (6 \cdot x) \cdot  \ln (6 \cdot x)} - 0$

Очевидно, что на данное выражение без слез не взглянешь. Авторы считают, что его необходимо упростить. Путем тривиальных преобразований получим:

$f'(x) = $ $\frac{\frac{5}{5 \cdot x - 6} \cdot  \ln (6 \cdot x) -  \ln (5 \cdot x - 6) \cdot \frac{6}{6 \cdot x}}{ \ln (6 \cdot x) \cdot  \ln (6 \cdot x)}$\section{Вычисление второй производной}

Легко понять, что вычисление второй производной -- дело не менее важное, чем вычисление первой. По определению вторая производная суть производная производной, исходя из чего некоторые авторы статьи посчитали, что читатель, дошедший до этого места, без труда вычислит вторую производную самостоятельно. Всё же, в ходе всенародного голосования единогласно было принято решение привести в данной статье нахождение второй производной заданной функции.

Очевидно, что

$6 \cdot x$ $\Rightarrow$ $0 \cdot x + 6 \cdot 1$

Хотите брать -- надо брать!

$ \ln (6 \cdot x)$ $\Rightarrow$ $\frac{0 \cdot x + 6 \cdot 1}{6 \cdot x}$

Вычислим производную данного выражения:

$6 \cdot x$ $\Rightarrow$ $0 \cdot x + 6 \cdot 1$

Что ещё уготовит на судьба?

$ \ln (6 \cdot x)$ $\Rightarrow$ $\frac{0 \cdot x + 6 \cdot 1}{6 \cdot x}$

Производящее хозяйство нынче не в моде. Произведем хотя бы функцию

$ \ln (6 \cdot x) \cdot  \ln (6 \cdot x)$ $\Rightarrow$ $\frac{0 \cdot x + 6 \cdot 1}{6 \cdot x} \cdot  \ln (6 \cdot x) +  \ln (6 \cdot x) \cdot \frac{0 \cdot x + 6 \cdot 1}{6 \cdot x}$

Что ещё уготовит на судьба?

$6 \cdot x$ $\Rightarrow$ $0 \cdot x + 6 \cdot 1$

Не все любят производную частного, но придется ее полюбить:

$\frac{6}{6 \cdot x}$ $\Rightarrow$ $\frac{0 \cdot 6 \cdot x - 6 \cdot (0 \cdot x + 6 \cdot 1)}{6 \cdot x \cdot 6 \cdot x}$

Вы что, не хотите брать производную? Что бы вы делали без авторов статьи:

$5 \cdot x$ $\Rightarrow$ $0 \cdot x + 5 \cdot 1$

Ещё в советских яслях проходили, что

$5 \cdot x - 6$ $\Rightarrow$ $(0 \cdot x + 5 \cdot 1) - 0$

Авторы статьи сомневаются в её необходимости. Это же и так очевидно:

$ \ln (5 \cdot x - 6)$ $\Rightarrow$ $\frac{(0 \cdot x + 5 \cdot 1) - 0}{5 \cdot x - 6}$

Конец близок, он где-то в этой производной:

$ \ln (5 \cdot x - 6) \cdot \frac{6}{6 \cdot x}$ $\Rightarrow$ $\frac{(0 \cdot x + 5 \cdot 1) - 0}{5 \cdot x - 6} \cdot \frac{6}{6 \cdot x} +  \ln (5 \cdot x - 6) \cdot \frac{0 \cdot 6 \cdot x - 6 \cdot (0 \cdot x + 6 \cdot 1)}{6 \cdot x \cdot 6 \cdot x}$

Производящее хозяйство нынче не в моде. Произведем хотя бы функцию

$6 \cdot x$ $\Rightarrow$ $0 \cdot x + 6 \cdot 1$

Производящее хозяйство нынче не в моде. Произведем хотя бы функцию

$ \ln (6 \cdot x)$ $\Rightarrow$ $\frac{0 \cdot x + 6 \cdot 1}{6 \cdot x}$

Производящее хозяйство нынче не в моде. Произведем хотя бы функцию

$5 \cdot x$ $\Rightarrow$ $0 \cdot x + 5 \cdot 1$

Любой школьник знает, что

$5 \cdot x - 6$ $\Rightarrow$ $(0 \cdot x + 5 \cdot 1) - 0$

Авторы статьи вопрошают: как посчитать эту производную? Ответ убил:

$\frac{5}{5 \cdot x - 6}$ $\Rightarrow$ $\frac{0 \cdot (5 \cdot x - 6) - 5 \cdot ((0 \cdot x + 5 \cdot 1) - 0)}{(5 \cdot x - 6) \cdot (5 \cdot x - 6)}$

Авторы статьи вопрошают: как посчитать эту производную? Ответ убил:

$\frac{5}{5 \cdot x - 6} \cdot  \ln (6 \cdot x)$ $\Rightarrow$ $\frac{0 \cdot (5 \cdot x - 6) - 5 \cdot ((0 \cdot x + 5 \cdot 1) - 0)}{(5 \cdot x - 6) \cdot (5 \cdot x - 6)} \cdot  \ln (6 \cdot x) + \frac{5}{5 \cdot x - 6} \cdot \frac{0 \cdot x + 6 \cdot 1}{6 \cdot x}$

Ещё в советских яслях проходили, что

$\frac{5}{5 \cdot x - 6} \cdot  \ln (6 \cdot x) -  \ln (5 \cdot x - 6) \cdot \frac{6}{6 \cdot x}$ $\Rightarrow$ $(\frac{0 \cdot (5 \cdot x - 6) - 5 \cdot ((0 \cdot x + 5 \cdot 1) - 0)}{(5 \cdot x - 6) \cdot (5 \cdot x - 6)} \cdot  \ln (6 \cdot x) + \frac{5}{5 \cdot x - 6} \cdot \frac{0 \cdot x + 6 \cdot 1}{6 \cdot x}) - (\frac{(0 \cdot x + 5 \cdot 1) - 0}{5 \cdot x - 6} \cdot \frac{6}{6 \cdot x} +  \ln (5 \cdot x - 6) \cdot \frac{0 \cdot 6 \cdot x - 6 \cdot (0 \cdot x + 6 \cdot 1)}{6 \cdot x \cdot 6 \cdot x})$

Любой школьник знает, что

$\frac{\frac{5}{5 \cdot x - 6} \cdot  \ln (6 \cdot x) -  \ln (5 \cdot x - 6) \cdot \frac{6}{6 \cdot x}}{ \ln (6 \cdot x) \cdot  \ln (6 \cdot x)}$ $\Rightarrow$ $\frac{((\frac{0 \cdot (5 \cdot x - 6) - 5 \cdot ((0 \cdot x + 5 \cdot 1) - 0)}{(5 \cdot x - 6) \cdot (5 \cdot x - 6)} \cdot  \ln (6 \cdot x) + \frac{5}{5 \cdot x - 6} \cdot \frac{0 \cdot x + 6 \cdot 1}{6 \cdot x}) - (\frac{(0 \cdot x + 5 \cdot 1) - 0}{5 \cdot x - 6} \cdot \frac{6}{6 \cdot x} +  \ln (5 \cdot x - 6) \cdot \frac{0 \cdot 6 \cdot x - 6 \cdot (0 \cdot x + 6 \cdot 1)}{6 \cdot x \cdot 6 \cdot x})) \cdot  \ln (6 \cdot x) \cdot  \ln (6 \cdot x) - (\frac{5}{5 \cdot x - 6} \cdot  \ln (6 \cdot x) -  \ln (5 \cdot x - 6) \cdot \frac{6}{6 \cdot x}) \cdot (\frac{0 \cdot x + 6 \cdot 1}{6 \cdot x} \cdot  \ln (6 \cdot x) +  \ln (6 \cdot x) \cdot \frac{0 \cdot x + 6 \cdot 1}{6 \cdot x})}{ \ln (6 \cdot x) \cdot  \ln (6 \cdot x) \cdot  \ln (6 \cdot x) \cdot  \ln (6 \cdot x)}$

С помощью нехитрых преобразований можем привести вторую производную к виду:

$f''(x) = $ $\frac{((\frac{-25}{(5 \cdot x - 6) \cdot (5 \cdot x - 6)} \cdot  \ln (6 \cdot x) + \frac{5}{5 \cdot x - 6} \cdot \frac{6}{6 \cdot x}) - (\frac{5}{5 \cdot x - 6} \cdot \frac{6}{6 \cdot x} +  \ln (5 \cdot x - 6) \cdot \frac{-36}{6 \cdot x \cdot 6 \cdot x})) \cdot  \ln (6 \cdot x) \cdot  \ln (6 \cdot x) - (\frac{5}{5 \cdot x - 6} \cdot  \ln (6 \cdot x) -  \ln (5 \cdot x - 6) \cdot \frac{6}{6 \cdot x}) \cdot (\frac{6}{6 \cdot x} \cdot  \ln (6 \cdot x) +  \ln (6 \cdot x) \cdot \frac{6}{6 \cdot x})}{ \ln (6 \cdot x) \cdot  \ln (6 \cdot x) \cdot  \ln (6 \cdot x) \cdot  \ln (6 \cdot x)}$\section*{Источники}
\begin{enumerate}
	\item Любовь к математике;
	\item Горящие сердца авторов статьи;
	\item Божественные знания, данные нам Матом и Аном.
\end{enumerate}

\end{document}

