\documentclass{article}
\usepackage[russian]{babel}
\usepackage{graphicx}

\title{Вычисление производной}
\author{Авторы статьи возжелали остаться в тени}


\begin{document}

\maketitle

\begin{abstract}
	Время - иллюзия. Бот бессмертен. Математика вечна. Аминь
\end{abstract}

\section{Вступление}

У современного человека в жизни часто возникает потребность дифференцировать. Бывает, что человек не может даже сопротивляться этому порыву: дай продифференцировать что-то или умру! Главная проблема заключается в том, что не каждый человек от рождения умеет вычислять производные функций. Авторы статьи решили помочь обездоленным и предоставить пример дифференцирования одной из функций, с помощью которого любой, даже самый отпетый гуманитарий, смог бы научиться брать простенькие производные.

Кроме того, данная статья поясняет одно из важных приложений производных -- исследование функции. Вычисление производных является необходимой и важнейшей задачей при исследовании функции и построении её графика. При известных первой и второй производных читатель с легкостью сможет построить график функции, а данная статья научить эти производные искать.

Несмотря на свои добрые побуждения, авторы статьи не могут знать, к чему приведет ее использование. Авторы статьи не несут ответсвенности за последствия. Всё на вашей совести.

\section{Вычисление первой производной}

Давайте разберем такой несложный пример. Рассмотрим функцию 

$f(x) = $ $ sin (x)$ и найдем ее производную.

Мы начнём с вычисления элементарных производных и постепенно дойдем до всей функции.

Вычислим производную данного выражения:

$ sin (x)$ $\Rightarrow$ $1 \cdot  cos (x)$

Очевидно, что на данное выражение без слез не взглянешь. Авторы считают, что его необходимо упростить. Путем тривиальных преобразований получим:

$f'(x) = $ $ cos (x)$

\section{Вычисление второй производной}

Легко понять, что вычисление второй производной -- дело не менее важное, чем вычисление первой. По определению вторая производная суть производная производной, исходя из чего некоторые авторы статьи посчитали, что читатель, дошедший до этого места, без труда вычислит вторую производную самостоятельно. Всё же, в ходе всенародного голосования единогласно было принято решение привести в данной статье нахождение второй производной заданной функции.

Ещё в советских яслях проходили, что

$ cos (x)$ $\Rightarrow$ $1 \cdot (-1) \cdot  sin (x)$

С помощью нехитрых преобразований можем привести вторую производную к виду:

$f''(x) = $ $(-1) \cdot  sin (x)$\section{Размышления об исследовании функции}

Теперь, когда вычислены первая и вторая производная, дальнейшие действия по исследованию функции тривиальны. Авторы статьи не считают необходимым приводить график заданной функции, ведь мудрый и увлеченный читатель уже сам давно его построил. Для тех же читателей, что еще не постоили график, эта задача предлагается в качестве упражнения.

Для закрепления знаний, связанных с нахождением производной, полученных в данной статье, авторы статьи предлагают читателю вычислить первую, вторую, третью, четвертую и пятую производные в точке $x_0$ в качестве упражнения.

Зная эти производные, можно получить разложение функции по формуле Тейлора в окрестности данной точки. Авторы статьи надеятся, что данное упражнение не вызовет у читателя затруднений.

Формула Тейлора имеет следующий вид:
$$f(x) = f(x_0) + \frac{f'(x_0)}{1!}(x - x_0) + \cdots + \frac{f^{(n)}(x_0)}{n!}(x - x_0)^n + o((x - x_0)^n).$$

Чтобы упростить читателю задачу, авторы статьи вычислили первую и вторую производную в некоторой точке $x_0$. Один из авторов статьи оказался очень сердобольным человеком и настоял на том, чтобы хотя бы третья производная присутствовала в статье в неоктором виде. Авторы статьи решили привести численное значениие этой производной в точке $x_0$ и ничего более, дабы не сокращать интерес читателя к дифференцированию и его любопытство. Читатель! Вполне возможно, что мы допустили ошибку. Надежда только на тебя!

\section{Разложение по формуле Тейлора и графики}

Теперь мы можем, ради затравочки, начертить простенький эскиз графика $f(x)$ и отметить на нем разложения по Тейлору вплоть до 3 степени. Читаель сам с легкостью, ориентируясь на эти примеры, начертит точный график функции и отметит разложение до 4 и 5 степеней. Ведь самое сложное в исследовании функции, а именно вычисление первой и второй производной, уже давно сделано, и с оставшейся задачей справится любой пятиклассник.



$x_0 = 0$

$f(x_0) = 0$

$f'(x_0) = 1$

$f''(x_0) = -0$

$f'''(x_0) = -1$

Итак, приступим:

$f(x) = x +o(x)$

\includegraphics[width=1\linewidth]{gnuplot/graph1.png}

$f(x) = x +o(x^2)$

\includegraphics[width=1\linewidth]{gnuplot/graph2.png}

$f(x) = x -0.166667x^3 +o(x^3)$

\includegraphics[width=1\linewidth]{gnuplot/graph3.png}

\section{Заключение}

Итак, подошла к конца наша общая и полная статья по дифференцированию и исследованию функции. Авторы статьи искренне благодарят за прочтение дошедших до этого момента и желают удачи в дифференцировании. Помните: дифференцируйте свои проблемы, а не интегрируйте! И да пребудет с вами математика

\section*{Источники}
\begin{enumerate}
	\item Любовь к математике;
	\item Горящие сердца авторов статьи;
	\item Божественные знания, данные нам Матом и Аном.
\end{enumerate}

\end{document}

